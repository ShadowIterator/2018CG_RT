\documentclass{article}
\usepackage{amsmath}
\usepackage{geometry}
\usepackage{CJK}
\usepackage{ctex}
\usepackage{graphicx}
\usepackage{booktabs}
%\usepackage{minted}
\usepackage{pst-node}
\usepackage{pst-text}
\usepackage{amsmath}
\usepackage{amssymb}
\usepackage{listings}
\usepackage{fancyhdr}
\usepackage[colorlinks]{hyperref}
\usepackage{color}
\definecolor{gray}{rgb}{0.4,0.4,0.4}
\definecolor{darkblue}{rgb}{0.0,0.0,0.6}
\definecolor{cyan}{rgb}{0.0,0.6,0.6}
\definecolor{grey}{rgb}{0.7,0.7,0.7}

\lstset{
  basicstyle=\ttfamily,
  columns=fullflexible,
  showstringspaces=false,
  commentstyle=\color{gray}\upshape
}
\lstdefinelanguage{XML}
{
  morestring=[b]",
  morestring=[s]{>}{<},
  morecomment=[s]{<?}{?>},
  stringstyle=\color{black},
  identifierstyle=\color{darkblue},
  keywordstyle=\color{cyan},
  morekeywords={xmlns,version,type}% list your attributes here
}

\title{光线跟踪大作业}
\author{李仁杰\\2016013271\\ShadowIterator@hotmail.com}
\date{}
\geometry{left=2cm,right=2cm,top=1cm,bottom=1cm}
\begin{document}
  \maketitle
  \section*{\color{blue}环境}
    \noindent
    \begin{itemize}
        \item Visual Studio 2015 professional
        \item C++
        \item opencv3.4.1
        \item Windows10-TH2-EDU
    \end{itemize}

    \section*{\color{blue}实现的算法}
    \subsection*{\color{gray}RayTracing}
    对应代码段在$raytracer.cpp$文件下的$Run$函数。\\
    反射折射对应的函数声明在$raytracer.h$文件下$RayTracer$类的$CalnDiffusion$、$CalnReflection$、$CalnRefraction$成员函数。\\
    阴影对应的函数声明在$light.h$文件下$Light$类的$CalnShade$成员函数。
    \subsection*{\color{gray}抗锯齿}
    对应代码段在$raytracer.cpp$文件下的$149$到$165$行。\\
    采用边缘超采样。\\
    判断边缘的方法:对于一个像素点,如果它周围的点发出的光线和它发出的光线碰撞到的物体(包括反射后碰撞到的物体)是一样的,则不是边缘,否则是边缘。
    \subsection*{\color{gray}纹理贴图}
    实现了平面和球面的纹理贴图。对应代码段在$primitive.cpp$文件下的$GetTexture$函数。
    \section*{\color{blue}文件路径}
        \subsubsection*{\color{black}源文件}
        \noindent
        \begin{lstlisting}
        \src\HW1\HW1\HW1.cpp
        \end{lstlisting}
        \subsubsection*{\color{black}可执行文件}
        \noindent
        \begin{lstlisting}
        \bin\HW1.exe
        \end{lstlisting}
        \subsubsection*{\color{black}导出图像}
        \noindent
        \begin{lstlisting}
        \bin\output.png
        \end{lstlisting}
  \newpage
  \section*{\color{blue}场景说明}
  \begin{itemize}
    \item 一个水平(平行于$XOY$平面)的地面,纹理是$floor.bmp$
    \item 一个与$ZOY$平面平行的“墙壁”,纹理是$wallpaper-2.jpg$
    \item 一个与$ZOX$平面平行的“镜子”
    \item 一个放置在地面上的红色透明玻璃球
    \item 一个悬浮在空中的无色透明玻璃球
    \item 一个放置在地面上的大理石纹理球,纹理是$marble-3.jpg$
    \item 一个悬浮在空中的蓝色反光正八面体
    \item 一个天上的与$XOY$平面平行的面光源
  \end{itemize}
  \section*{\color{blue}参考资料}
  主要参考了计33古裔正学长的代码。 
\end{document}
